\documentclass[10pt,a4paper]{article}

% параметры книги, заданные в издательстве
\usepackage[top=30mm, left=20mm, right=20mm]{geometry}

% шрифты для XeLaTeX
\usepackage{fontspec}
\setmainfont{DejaVu Serif}
\setsansfont{DejaVu Sans}
\setmonofont{DejaVu Sans Mono}

% параграфы без красной строки, расстояние между параграфами 4 пункта
\setlength{\parindent}{0pt}
\setlength{\parskip}{4pt}
            
% поддержка типографских соглашений и переносов для английского и русского языков
\usepackage[english, russian]{babel}

% рисунки в формате PDF в том же каталоге
\usepackage{graphicx}

% относительные размеры текста
\usepackage{relsize}


\begin{document}

\section*{Смертельный беклог}

Компрект: счётные палочки (100 штук), шестигранный кубик с числами -2, -1, 0, 1, 2 и 3 на сторонах.

Складываем палочки кучкой в центре.
Сначала каждый игрок перекладывает себе по шесть палочек.
Каждая палочка олицетворяет одну задачу из беклога. Палочки, которые оказались у игрока — это сделанные задачи.

Игроки по-очереди бросают кубик.
Если выпадает -2 или -1, игрок возвращает две или одну палочку в общую кучу.
Нашли ошибку в сделанной задаче и вернули на исправление.

Если на кубике выпадает 0, игрок пропускает ход.
Бизнес взял тайм-аут, чтобы подумать над задачами.

Если выпадает 1, 2 или 3, игрок забирает из общей кучи одну, две или три палочки в свою кучу.
Команда хорошо поработала и задачи были без подводных команей.
В общем, повезло.

Игра заканчивается, когда заканчиваются палочки в общей куче.
Побеждает игрок, у которого больше всех палочек или, иными словами, тот, кто «сделал» больше всех задач.

Если у кого-то из игроков заканчиваются палочки, он выбывает из игры.
Команда не сделала минимально жизнеспособный продукт (Minimum Viable Product, MVP) в оговоренные сроки и проект закрыли.

\end{document}
