\documentclass[11pt,a4paper]{article}

\usepackage{fontspec}
\setmainfont{DejaVu Serif}
\setsansfont{DejaVu Sans}

% параметры книги, заданные в издательстве
\usepackage[left=20mm, right=20mm]{geometry}

% параграфы без красной строки, расстояние между параграфами 4 пункта
\setlength{\parindent}{0pt}
\setlength{\parskip}{4pt}
            
% поддержка типографских соглашений и переносов для английского и русского языков
\usepackage[english, russian]{babel}

% шрифты
\usepackage[default]{droidserif}
\usepackage[defaultsans]{droidsans}

% рисунки в формате PDF в том же каталоге
\usepackage{graphicx}


\begin{document}

\thispagestyle{empty}

\section*{Смертельный беклог \\
{\smaller Deadlog}}

Комплект:

\begin{itemize}
    \item фишки, 100 штук;
    \item кубик с числами -2, -1, 0, 1, 2 и 3 на гранях.
\end{itemize}

\subsection*{Правила}

Складываем фишки кучкой в центре — это беклог.
Каждая фишка олицетворяет одну задачу.
Перед началом игры игроки перекладывают себе по шесть фишек из беклога — задачи, которые они сделали.

Далее игроки по-очереди бросают кубик.

\begin{itemize}
    \item Если выпадает -2 или -1, игрок возвращает две или одну фишку в беклог.
    \textit{Нашли ошибку в сделанной задаче и вернули на исправление.}
    Если у игрока не хватает фишек, чтобы вернуть в беклог, он выбывает из игры.
    \textbf{Важно:} если игрок вернул фишки и остался с пустыми руками, он продолжает игру.
    Игрок выбывает, только если он ушёл «в минус».
    \textit{Работал спустя рукава, много косячил — вот его и уволили.}

    \item Если на кубике выпадает 0, игрок пропускает ход.
    \textit{Бизнес взял тайм-аут, чтобы подумать над задачами.}

    \item Если выпадает 1, 2 или 3, игрок забирает из общей кучи одну, две или три фишки в свою кучу.
    \textit{Команда хорошо поработала и задачи были без подводных команей. В общем, повезло.}
\end{itemize}

Игра заканчивается, когда заканчиваются задачи в беклоге.
Побеждает игрок, у которого больше всех фишек, то есть тот, кто сделал больше всех задач.

\subsection*{Риск и страховка}

Игрок может рискнуть или подстраховаться перед тем, как бросить кубик.

\begin{itemize}
    \item Рискующий игрок перед броском предупреждает всех, что он — оптимист.
    Если кубик выпадает зелёной (оптимистичной) гранью с числами 1 или 2, игрок забирает из беклога на одну фишку больше.
    \textit{Награда за риск.}
    Если кубик выпадает красной (пессимистичной) гранью с числами -2 или -1, игрок возвращает в беклог на одну фишку больше.
    \textit{Наказание за безумства.}

    \item Страхующихся игрок перед броском сообщает всем, что он — пессимист.
    Если кубик выпадает красной (пессимистичной) гранью, игрок возвращает в беклог на одну фишку меньше.
    Если выпало число -1, игрок не возвращает ничего и остаётся «при своих».
    \textit{Награда за предусмотрительность.}
    Если кубик выпадает зелёной (оптимистичной) гранью, игрок забирает из беклога на одну фишку меньше.
    \textit{Кто не рискует — тот не пьёт шампанского!}

    \item Если игрок не рискует и не подстраховывается, или если на кубике выпала синяя грань с числами 0 или 1, игра идёт по обычным правилам.
\end{itemize}

\center{\copyright Московский клуб программистов, 2004--2005 \includegraphics[trim=0 2mm 0 0]{progmsk}}

\end{document}
