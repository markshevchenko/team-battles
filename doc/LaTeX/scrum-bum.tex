\documentclass{article}
\usepackage[russian]{babel}
\usepackage{array}
\usepackage{fdsymbol}
\usepackage{xcolor}
\usepackage{ragged2e}

\let\mathdollar\undefined

\usepackage{fontspec}
\setmainfont{Noto Serif}
\setsansfont{Noto Sans}
\setmonofont{Noto Sans Mono}

\begin{document}

\section*{Скрам-Бам!}

\justifying

Комплект: карты игральные 36 штук, игральный кубик, разноцветные фишки.
Раздаём по 20 фишек каждому играющему~--- это полный объём беклога (запас).
Каждый игрок перекладывает себе по 6 фишек~--- это его актуальный беклог.
Каждая фишка олицетворяет задачу на один стори-поинт.

На каждом шаге игрок бросает кубик. 1, 2 или 3 очка означают, что владелец продукта приготовил 1, 2 или 3 истории для беклока. 4 очка означают, что владелец не приготовли историй, 5 очков~--- одна история из беклога отменяется, 6 очков~--- бизнес так и не смог договориться, непонятно, что делать дальше, игрок пропускает ход. Игрок перекладывает из запаса в актуальный беклог нужное количество фишек, а если было 5 очков, то перекладывает одну фишку из беклога в сделанные задачи.

Пополнив беклог, игрок выбирает одну игральную карту и делает то, что описано в таблице ниже. Если команда выполняет один, два или три стори-поинта, то нужное количество фишек перекладывается в сделанные задачи.

Игра заканчивается тогда, когда кто-то из игроков израсходовал весь запас, то есть полностью сделал проект, либо если закончились игральные карты. В последнем случае побеждает тот, кто сделал больше всех задач.

\begin{center}
\begin{tabular}{ | c | m{2.8cm} | m{2.8cm} | m{2.8cm} | m{2.8cm} | }
\hline
 & {\huge $\clubsuit$} & {\huge\color{red} $\vardiamondsuit$} & {\huge $\spadesuit$} & {\huge\color{red} $\varheartsuit$} \\
\hline
 6 & 1 SP & 3 SP & 2 SP & 0 SP \\
\hline
 7 & 2 SP & 4 SP & -1 SP & 1 SP \\
\hline
 8 & 0 SP & 2 SP & 1 SP & 3 SP \\
\hline
 9 & -2 SP & 1 SP & 0 SP & -1 SP \\
\hline
10 & 2 SP & 3 SP & 1 SP & -3 SP \\
\hline
 J & 1 SP & -2 SP & 0 SP & 0 SP \\
\hline
 Q & -1 SP & 2 SP & -4 SP & 1 SP \\
\hline
 K & -2 SP & 1 SP & 3 SP & 4 SP \\
\hline
 A & 0 SP & -3 SP & -1 SP & 2 SP \\
 \hline
\end{tabular}
\end{center}

1 SP~--- нормальная ситуация, команда сожгла за спринт задач на 1 стори-поинт.

2 SP, 3 SP, 4 SP~--- удачные спринты, когда команда сжигает 2, 3 или 4 стори-поинта.

0 SP~--- задача оказалась сложнее, чем думали. Не успели сделать, команда ничего не сожгла за спринт.

-1 SP, -2 SP, -3 SP и -4 SP~--- в сделанных задачах обнаружили ошибки и вернули их в виде багов. Команда будем заниматься ими. Нужное количество фишек переложить из сделанных задач в актуальный беклог.

\end{document}
