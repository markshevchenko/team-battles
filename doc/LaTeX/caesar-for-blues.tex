\documentclass[a4paper,10pt]{article}
\usepackage[russian]{babel}
\usepackage{fontspec}
\usepackage{nopageno}
\setmainfont{DejaVu Serif}

\begin{document}

\section*{Шифр Цезаря\\
(задача для синих~--- декодирование)}

Реализовать дешифрование по методу Цезаря.

Рассказывают, что Цезарь придумал простой шифр для латинских текстов, позволяющий зашифровывать и расшифровывать их практически "на лету".

Для шифрования нужно выбрать число $N$ и вместо каждой исходной буквы алфавита записывать $N$-ую букву справа.
Если вы выходите за границы алфавита, просто продолжайте считать буквы с левого края.

\begin{center}
A B C D E F G H I J K L M N O P Q R S T U V W X Y Z
\end{center}

Расшифровка делается точно также, только буквы следует отсчитывать не вправо, а влево и, добравшись до начала алфавита, надо продолжать отсчёт с конца.

При $N = 5$ слово \textbf{BNXITR} должно превратиться в \textbf{WISDOM}.

\subsection*{Вход программы}

Программа получает на вход две строки, содержащие только печатные символы ASCII.
В первой строке находится число $N$, $0 \leq N \leq 100$, во второй~--- строка для декодирования.

\noindent\textbf{Пример входа программы}\\
\texttt{
5\\
Xujfp Kwnjsi fsi Jsyjw
}

Программа должна декодировать все символы строки следующим образом:

\begin{itemize}
    \item Если символ~--- большая буква английского алфавита, перевести её в большую букву.
    \item Если символ~--- маленькая буква английского алфавита, перевести её в маленьку букву.
    \item Любой другой символ оставить без изменений.
\end{itemize}

\noindent\textbf{Пример выхода программы}\\
\texttt{
Speak Friend and Enter
}

\end{document}