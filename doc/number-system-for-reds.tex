\documentclass[a4paper,10pt]{article}
\usepackage[russian]{babel}
\usepackage{fontspec}
\usepackage{nopageno}
\setmainfont{DejaVu Serif}

\begin{document}

\section*{Система счисления\\
(задача для красных~--- кодирование)}

Реализовать перевод десятичный чисел в систему счисления, описанную ниже.

Система счисления состоит из цифр \textbf{0}, \textbf{1}, \textbf{2}, \textbf{3}, \textbf{4} и букв \textbf{a}, \textbf{b}, \textbf{c}, \textbf{d}, \textbf{e}, которым соответствуют \textit{отрицательные величины} -1, -2, -3, -4 и -5.

Число \textbf{1a} означает $1 * 10 + (-1) * 1 = 9$.

Число \textbf{2bc} означает $2 * 100 + (-2) * 10 + (-3) * 1 = 177$. 

\subsection*{Вход программы}

Программа получает на вход число в десятичной системе счисления $N$, $1 \leq N \leq 2\,000\,000\,000$.

\noindent\textbf{Пример входа программы}\\
\texttt{177}

Программа должна перевести десятичное число в систему, описанную выше и напечатать результат.

\noindent\textbf{Пример выхода программы}\\
\texttt{2bc}

\end{document}